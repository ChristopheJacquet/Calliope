%!TEX encoding = UTF-8 Unicode

\documentclass{beamer}
\usetheme{Singapore}

\usepackage[francais]{babel}

\usepackage[utf8]{luainputenc}
\usepackage[T1]{fontenc}
\usepackage{textcomp}

%\usepackage{xcolor}
%\usepackage{luacolor}


\usepackage{tikz}
\usetikzlibrary{automata, positioning}



\newcommand{\lettre}[1]{\emph{\structure{#1}}}
\newcommand{\latin}[1]{\structure{#1}}
\newcommand{\crx}{$\times$}

\newcommand{\syll}[1]{\fbox{#1 \hspace{5mm}}}

\definecolor{diacrit}{rgb}{.85,0,0}

\newcommand\overtxt[2]{\makebox[0cm][l]{\color{diacrit}#1}#2} 

\newcommand\hlbr{\overtxt{\u~}}
\newcommand\hllg{\overtxt{\=~}}

\setbeamertemplate{navigation symbols}{}


\title{\mbox{ Vers un outil pédagogique d'aide à la scansion latine} }
\author{Christophe Jacquet\\
{\footnotesize Supélec, Département informatique}
}
\date{8 février 2013}


\begin{document}

\maketitle


\begin{frame}{Présentation du projet}

\begin{itemize}
\item La scansion latine: position du problème

\begin{itemize}
\item un vers est décomposé en syllabes,\\ dont chacune peut être longue ou brève
\item un type de poème impose un ensemble de schémas possibles pour un vers, en termes de longueurs de syllabes
\item \structure{scander un vers}: l'aligner sur l'un des schémas possibles
\end{itemize}

\vfill

\item Objectif:

\begin{itemize}
\item Explorer l'aide informatique à la scansion
\item Un outil entièrement automatique semble inaccessible
\item Outil vu comme une aide pédagogique
\end{itemize}

\end{itemize}

\end{frame} %%%%%%%%%%%%%%%%%%%%%%%%%%%%%%%%%%%%%%%%%%%%%%%%%%%%%%%%%%


\begin{frame}{Plan}

\tableofcontents

\end{frame} %%%%%%%%%%%%%%%%%%%%%%%%%%%%%%%%%%%%%%%%%%%%%%%%%%%%%%%%%%


\section{Méthode de scansion}



\begin{frame}{Principe général}

Deux étapes:

\begin{enumerate}
\item Pré-traitement du vers:
\begin{itemize}
\item découper le vers en syllabes
\item contraindre au maximum les longueurs de syllabes qui ne souffrent (quasiment) pas d'exceptions
\end{itemize}

\item Alignement avec un schéma de vers

\begin{itemize}
\item On suppose connu le type de vers; on détermine alors les longueurs manquantes
\item Souvent plusieurs solutions
\item Parfois aucune...
\end{itemize}

\end{enumerate}
\end{frame} %%%%%%%%%%%%%%%%%%%%%%%%%%%%%%%%%%%%%%%%%%%%%%%%%%%%%%%%%%



\subsection{Pré-traitement du vers}

\subsubsection{Recherche des mots dans le dictionnaire}

\begin{frame}{Recherche des mots dans le dictionnaire}  %Détermination des longueurs par nature}

\begin{itemize}
\item Même en omettant cette étape, on obtient déjà des résultats intéressants!

\vfill

\item Ce que peut indiquer un dictionnaire:

\begin{itemize}
\item une voyelle longue ou brève \structure{par nature}
\item une voyelle en hiatus devant une autre \\
ex: Collatinus indique \latin{m{\u e}us} vs \latin{seu}
\item la valeur consonne ou voyelle de u/v et i/j
\end{itemize}

\vfill

\item Quel dictionnaire?

\begin{tabular}{r c c c}
				& Longues & Brèves & u/v, i/j \\
\hline
Gaffiot (Collatinus)		& \crx    & \crx   &  \\
Pocket Oxfort Latin & \crx &       & \crx \\
Lewis \& Short	& \crx    & \crx   & \crx \\
\hline
\end{tabular}


\end{itemize}

\end{frame} %%%%%%%%%%%%%%%%%%%%%%%%%%%%%%%%%%%%%%%%%%%%%%%%%%%%%%%%%%


\begin{frame}{Méthode employée pour la recherche de mots}

\begin{itemize}

\item Enjeu:

\begin{itemize}
\item pour les verbes, noms, adjectifs et pronoms, le dictionnaire donne le radical et le modèle de flexion
\item or dans un texte, on rencontre des formes fléchies
\end{itemize}

\vfill

\item[$\Rightarrow$] Génération de toutes les formes de la langue latine

\begin{itemize}
\item Utilisation du générateur de Philippe Verkerk pour le \emph{Pocket Oxford}
\item Passage de 10\,700 entrées à 615\,000 formes
\item Stockage dans une base de données pour un accès rapide
\item Taille raisonnable: 33 Mo
\end{itemize}

\end{itemize}

\end{frame} %%%%%%%%%%%%%%%%%%%%%%%%%%%%%%%%%%%%%%%%%%%%%%%%%%%%%%%%%%


\subsubsection{Découpage en syllabes}

\begin{frame}{Découpage en syllabes}

\begin{itemize}
\item Analyse du texte lettre par lettre
\item Principe général:\\ «~une syllabe = une voyelle suivie de consonnes~»
\item Règles:

\begin{itemize}
\item une diphtongue (\lettre{ae}, \lettre{oe}, \lettre{eu}, \lettre{au}) compte pour une seule voyelle (longue)
\item ... sauf si la première voyelle est marquée brève (hiatus)
\item \lettre{u} dans \lettre{qu} ne compte pas comme voyelle
\item une voyelle en fin de mot avant une voyelle s'élide
\end{itemize}

\end{itemize}
\end{frame} %%%%%%%%%%%%%%%%%%%%%%%%%%%%%%%%%%%%%%%%%%%%%%%%%%%%%%%%%%


\subsubsection{Longueurs par position}

\begin{frame}{Longueurs par position}

\begin{itemize}
\item Une voyelle sans longueur par nature, suivie de deux consonnes, doit être allongée
\item Quelques exceptions
\vfill
\item Si on utilise le dictionnaire pour obtenir les longueurs par nature, on contraint déjà beaucoup le problème\\
$\Rightarrow$ La détermination des longues par position peut être désactivée
\end{itemize}
\end{frame} %%%%%%%%%%%%%%%%%%%%%%%%%%%%%%%%%%%%%%%%%%%%%%%%%%%%%%%%%%


\subsection{Alignement avec un schéma de vers}

\begin{frame}{Alignement avec un schéma de vers}
\small

Exemple d'\structure{hexamètre dactylique} (Ovide, \emph{Métamorphoses}):

\centerline{\latin{Spem sine corpore amat, corpus putat esse quod unda est.}}

\begin{itemize}

\item
Première étape (découpage en pieds et contraintes de longueur):

\mbox{\footnotesize\latin{
spem s/in/e c/\hllg{o}rp/or/e am/at, c/\hllg{o}rp/us p/ut/at /\hllg{e}ss/e qu/od /\hllg{u}nd/a \hllg{e}st.
}}

\item
Schéma de l'hexamètre:\\ 
{\footnotesize Cinq dactyles ({\color{diacrit}\=\,\u\,\u\,\!}) ou spondées ({\color{diacrit}\=\,\=\,\!}), puis un spondée ou trochée ({\color{diacrit}\=\,\u\,\!}).
}

\item
Essai de toutes les combinaisons possibles à partir du début \\
{\footnotesize Il y en a ici $2^5 \times 2 = 64$ (très raisonnable)}

\item Une seule combinaison conforme aux contraintes:

\mbox{\footnotesize\latin{
sp\hllg{e}m s/\hlbr{\,\i}n/\hlbr{e} c/\hllg{o}rp/\hlbr{o}r/e \hlbr{a}m/\hllg{a}t, c/\hllg{o}rp/\hllg{u}s p/\hlbr{u}t/\hlbr{a}t /\hllg{e}ss/\hlbr{e} qu/\hlbr{o}d /\hllg{u}nd/a est.
}}

\end{itemize}
%Algorithmiquement réaliste: pour un vers de 20 syllabes, la limite absolue du nombre de possibilités est $2^{20} \approx 1\,000\,000$.

\end{frame} %%%%%%%%%%%%%%%%%%%%%%%%%%%%%%%%%%%%%%%%%%%%%%%%%%%%%%%%%%


\begin{frame}{Généralisation et formalisation}

\begin{itemize}

\item
Chaque type de vers est décrit «~formellement~» (de façon déclarative)\\

Exemple: pentamètre

\vspace{3mm}

\begin{tikzpicture}[->, node distance=1cm]
\tikzstyle{div}=[circle,thick,draw=black,fill=black,inner sep=0pt,minimum size=.5mm]
\tikzstyle{pied}=[rectangle,draw=black,fill=white]

\node[div] (d0) {};

\node[pied] (p1h) [above right=.3cm and .5cm of d0] {\=~\u~\u~};
\node[pied] (p1b) [below right=.3cm and .65cm of d0] {\=~\=~};
\node[div] (d1) [below right=.3cm and .5cm of p1h] {};

\node[pied] (p2h) [above right=.3cm and .5cm of d1] {\=~\u~\u~};
\node[pied] (p2b) [below right=.3cm and .65cm of d1] {\=~\=~};
\node[div] (d2) [below right=.3cm and .5cm of p2h] {};

\node[pied] (p3) [right=.2cm of d2] {\=~};

\node[pied] (p4) [right=.6cm of p3] {\=~\u~\u~};

\node[pied] (p5) [right=.6cm of p4] {\=~\u~\u~};

\node[pied] (p6) [right=.5cm of p5] {\u-};

\path (d0) edge (p1h.west);
\path (d0) edge (p1b.west);

\path (p1h.east) edge (d1);
\path (p1b.east) edge (d1);

\path (d1) edge (p2h.west);
\path (d1) edge (p2b.west);

\path (p2h.east) edge (d2);
\path (p2b.east) edge (d2);

\path (d2) edge (p3);
\path (p3) edge (p4);
\path (p4) edge (p5);
\path (p5) edge (p6);

\end{tikzpicture}

Pour le pentamètre $2 \times 2 \times 1 \times 1 \times 1 \times 1 = 4$ possibilités.

\vfill

\item Stratégie:

\begin{itemize}
\item essai exhaustif des combinaisons possibles \\
{\footnotesize (nombre toujours raisonnable)}
\item abandon d'une combinaison dès qu'elle ne respecte pas les contraintes a priori
\end{itemize}

\end{itemize}

\end{frame} %%%%%%%%%%%%%%%%%%%%%%%%%%%%%%%%%%%%%%%%%%%%%%%%%%%%%%%%%%





\section{Aspects techniques}

\subsection*{}

\begin{frame}[fragile]
\frametitle{Solutions techniques}

%\frametitle{toto}

\begin{itemize}
\item Langage de programmation: Python
\item Moteur de bases de données: SQLite

\vfill

\item Programme conçu pour être extensible\\
Exemple: ajout facile de nouveaux types de vers

{
\footnotesize
\begin{verbatim}
V_HEXAMETRE = TypeVers("hd", [
    ChoixPied([P_DACTYLE, P_SPONDEE]),
    ChoixPied([P_DACTYLE, P_SPONDEE]),
    ChoixPied([P_DACTYLE, P_SPONDEE]),
    ChoixPied([P_DACTYLE, P_SPONDEE]),
    ChoixPied([P_DACTYLE, P_SPONDEE]),
    ChoixPied([P_TROCHEE, P_SPONDEE]),
])
\end{verbatim}
}

\vfill

\item Accès via une interface web

\end{itemize}
\end{frame} 
%%%%%%%%%%%%%%%%%%%%%%%%%%%%%%%%%%%%%%%%%%%%%%%%%%%%%%%%%%


\begin{frame}{Diffusion et mise à disposition}

\begin{itemize}
\item Mode de diffusion

\begin{itemize}
\item code: GNU General Public License (GPL)
\item dictionnaires:

\begin{itemize}
\item Gaffiot de 1934: dans le domaine public depuis 2007 (70 ans après la mort de Félix Gaffiot)
\item Pocket Oxford Latin?
\item Lewis \& Short de 1879: idem Gaffiot
\end{itemize}


\end{itemize}

\item Développement?

\item Hébergement de l'outil?

\end{itemize}
\end{frame} %%%%%%%%%%%%%%%%%%%%%%%%%%%%%%%%%%%%%%%%%%%%%%%%%%%%%%%%%%



\section{Limites et perspectives}

\subsection*{}


\begin{frame}{Limites actuelles du démonstrateur, \\ améliorations projetées}

\begin{itemize}
\item \structure{Dictionnaire:} actuellement seules les longues sont indiquées\\
$\Rightarrow$ Intégration d'un dictionnaire plus complet: Lewis \& Short?

\vfill

\item Non-détection des \structure{voyelles en hiatus} {\small (découle du point précédent)}

\vfill

\item Seules quelques \structure{règles simples} prises en compte actuellement\\
$\Rightarrow$ Ajouter des règles?\\
\hfill Compromis entre exhaustivité et lisibilité du programme

\end{itemize}
\end{frame} %%%%%%%%%%%%%%%%%%%%%%%%%%%%%%%%%%%%%%%%%%%%%%%%%%%%%%%%%%


\begin{frame}{Limites inhérentes à ce type d'outil}

\begin{itemize}
\item \structure{Homographes}\\
Exemple: p\hlbr{o}p\u{u}lus vs p\hllg{o}p\u{u}lus \\
$\Rightarrow$ le dictionnaire ne doit contenir que l'indication pop\u{u}lus

\vfill

\item \structure{Exceptions rares:} difficiles à prendre en compte sous peine de sous-contraindre la solution

\vfill

\item \structure{Règles enfreintes par l'auteur:} comment privilégier le non-respect d'une règle plutôt qu'une autre?

\vfill

\item \structure{Type de vers non précisé:} essayer de réaliser l'alignement avec tous les types de vers connus? Explosion du nombre de propositions?

\end{itemize}
\end{frame} %%%%%%%%%%%%%%%%%%%%%%%%%%%%%%%%%%%%%%%%%%%%%%%%%%%%%%%%%%




\begin{frame}{Intérêt pédagogique actuel}

Même si l'outil ne donne pas automatiquement une solution unique à coup sûr, il fait des propositions qui peuvent guider l'étudiant:

\begin{itemize}
\item les longueurs par nature
\item les longueurs par position
\item une décomposition en syllabes \emph{a priori}
\end{itemize}

\vfill

Un enseignant peut exploiter les limites de l'outil pour montrer les chausse-trappes de l'exercice.

\end{frame} %%%%%%%%%%%%%%%%%%%%%%%%%%%%%%%%%%%%%%%%%%%%%%%%%%%%%%%%%%



\begin{frame}{Perspective: un outil interactif?}

Plutôt que proposer directement des résultats, un outil pédagogique pourrait engager un \structure{dialogue} avec l'élève:

\begin{itemize}
\item donner différentes propositions pour les homographes (cf. p\hlbr{o}p\u{u}lus/p\hllg{o}p\u{u}lus) avec les définitions

\item demander à l'élève de confirmer ou d'infirmer chaque choix: indication des longueurs, découpage en syllabes

\item montrer les implications qu'une ébauche de scansion a sur le sens du texte:\\ par exemple, cas déduit d'une longueur de désinence

\end{itemize}

\vfill

Cela permettrait de \structure{faire des hypothèses}, d'en voir rapidement les \structure{conséquences}, et de \structure{revenir en arrière} le cas échéant.

\end{frame} %%%%%%%%%%%%%%%%%%%%%%%%%%%%%%%%%%%%%%%%%%%%%%%%%%%%%%%%%%



\section*{}

\begin{frame}{Résumé}

\begin{itemize}
\item Un outil simple d'aide à la scansion
\item Limité mais extensible
\item Développement en un vrai outil pédagogique?
\end{itemize}

\vfill

\uncover<2->{
\centerline{\structure{Merci de votre attention}}
}

\vfill

\end{frame} %%%%%%%%%%%%%%%%%%%%%%%%%%%%%%%%%%%%%%%%%%%%%%%%%%%%%%%%%%

\end{document}